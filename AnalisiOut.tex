\hypertarget{calibrazione-dellelettronica-tramite-segnali-di-ampiezza-nota}{%
\section{Calibrazione dell'elettronica tramite segnali di ampiezza
nota}\label{calibrazione-dellelettronica-tramite-segnali-di-ampiezza-nota}}

\begin{longtable}[]{@{}rrrrrr@{}}
\toprule
ch1 & ch2 & delta & CHN & CNT & errore\_ch \\
\midrule
\endhead
0.968 & 0.992 & 24 & 195 & 1237 & 1 \\
1.98 & 2.06 & 80 & 401 & 996 & 1 \\
2.92 & 3.02 & 100 & 596 & 651 & 1 \\
3.92 & 4.04 & 120 & 793 & 768 & 1 \\
4.84 & 5 & 160 & 981 & 980 & 1 \\
5.88 & 6.04 & 160 & 1191 & 896 & 1 \\
6.92 & 7.08 & 160 & 1403 & 755 & 1 \\
7.76 & 8.08 & 320 & 1587 & 607 & 1 \\
8.8 & 9.12 & 320 & 1807 & 522 & 1 \\
9.2 & 9.52 & 320 & 1889 & 839 & 1 \\
\bottomrule
\end{longtable}

Il fit ha forma V = a*Ch+b. Il grafico è AX1 
\$a = 0.00498 \pm 0.00001\$ 
\$b = 0.00953 \pm 0.00429 \$

\begin{verbatim}
[[Fit Statistics]]
    # fitting method   = leastsq
    # function evals   = 7
    # data points      = 10
    # variables        = 2
    chi-square         = 0.03872345
    reduced chi-square = 0.00484043
    Akaike info crit   = -51.5389496
    Bayesian info crit = -50.9337795
[[Variables]]
    a:  0.00498374 +/- 9.7152e-06 (0.19%) (init = 1)
    b:  0.00953102 +/- 0.00429480 (45.06%) (init = 1)
[[Correlations]] (unreported correlations are < 0.100)
    C(a, b) = -0.724 
\end{verbatim}

\hypertarget{calibrazione-dellelettronica-tramite-segnali-di-ampiezza-nota-1}{%
\section{Calibrazione dell'elettronica tramite segnali di ampiezza
nota}\label{calibrazione-dellelettronica-tramite-segnali-di-ampiezza-nota-1}}

\begin{longtable}[]{@{}rrrrr@{}}
\toprule
Picco & CNT & MezzoPicco & FWHM & err \\
\midrule
\endhead
942 & 276 & 944 & 4 & 1.702 \\
1081 & 357 & 1083 & 4 & 1.702 \\
1144 & 312 & 1146 & 4 & 1.702 \\
\bottomrule
\end{longtable}

Il fit ha forma E\_note = a*Ch\_picco+b. Il grafico è AX2 
\$a = 0.00503\pm 0.00001 \$
 \$b = 0.04703 \pm 0.01050 \$

\begin{verbatim}
[[Fit Statistics]]
    # fitting method   = leastsq
    # function evals   = 7
    # data points      = 3
    # variables        = 2
    chi-square         = 2.10142915
    reduced chi-square = 2.10142915
    Akaike info crit   = 2.93201612
    Bayesian info crit = 1.12924070
[[Variables]]
    a:  0.00503250 +/- 9.9177e-06 (0.20%) (init = 1)
    b:  0.04702507 +/- 0.01050314 (22.34%) (init = 1)
[[Correlations]] (unreported correlations are < 0.100)
    C(a, b) = -0.997 
\end{verbatim}

\hypertarget{verifica-della-correttezza-della-retta-di-calibrazione-tramite-studio-dei-picchi-secondari}{%
\section{Verifica della correttezza della retta di calibrazione tramite
studio dei picchi
secondari}\label{verifica-della-correttezza-della-retta-di-calibrazione-tramite-studio-dei-picchi-secondari}}

\begin{longtable}[]{@{}
  >{\raggedright\arraybackslash}p{(\columnwidth - 8\tabcolsep) * \real{0.1617}}
  >{\raggedright\arraybackslash}p{(\columnwidth - 8\tabcolsep) * \real{0.1617}}
  >{\raggedright\arraybackslash}p{(\columnwidth - 8\tabcolsep) * \real{0.1677}}
  >{\raggedright\arraybackslash}p{(\columnwidth - 8\tabcolsep) * \real{0.2216}}
  >{\raggedleft\arraybackslash}p{(\columnwidth - 8\tabcolsep) * \real{0.2874}}@{}}
\toprule
\begin{minipage}[b]{\linewidth}\raggedright
Elemento e numero picco
\end{minipage} & \begin{minipage}[b]{\linewidth}\raggedright
Canali picchi secondari
\end{minipage} & \begin{minipage}[b]{\linewidth}\raggedright
Energie picchi secondari
\end{minipage} & \begin{minipage}[b]{\linewidth}\raggedright
Energie picchi secondari teoriche
\end{minipage} & \begin{minipage}[b]{\linewidth}\raggedleft
z = (Eteo - Esperim)/sqrt(u\_teo\^{}2 + u\_esp\^{}2)
\end{minipage} \\
\midrule
\endhead
Np III & 912.0+/-1.7 & 4.637+/-0.009 & 4.6390+/-0.0010 & 0.266014 \\
Np II & 939.0+/-1.7 & 4.773+/-0.009 & 4.7710+/-0.0010 & -0.176385 \\
Am III & 1062.0+/-1.7 & 5.392+/-0.009 & 5.3880+/-0.0010 & -0.408479 \\
Am II & 1072.0+/-1.7 & 5.442+/-0.009 & 5.4430+/-0.0010 & 0.131077 \\
Cm II & 1136.0+/-1.7 & 5.764+/-0.009 & 5.7630+/-0.0010 & -0.10851 \\
\bottomrule
\end{longtable}

Dove la terza riga è stata ottenuta passando la seconda dentro la retta
di calibrazione

\hypertarget{conteggi-totali-in-funzione-della-pressione}{%
\section{Conteggi totali in funzione della
pressione}\label{conteggi-totali-in-funzione-della-pressione}}

\begin{longtable}[]{@{}rr@{}}
\toprule
Pressione & Conteggi \\
\midrule
\endhead
0 & 4967 \\
205 & 4884 \\
403 & 4782 \\
598 & 4741 \\
612 & 4626 \\
621 & 4538 \\
631 & 4583 \\
641 & 4437 \\
650 & 4447 \\
670 & 4310 \\
685 & 3922 \\
690 & 3409 \\
695 & 2790 \\
702 & 1105 \\
750 & 15 \\
\bottomrule
\end{longtable}

Si esegue il fit utilizzando solo i punti:
 \textbar{} Pressione \textbar{} Conteggi \textbar{}
\textbar------------:\textbar-----------:\textbar{} \textbar{} 695
\textbar{} 2790 \textbar{} \textbar{} 702 \textbar{} 1105 \textbar{}

Il fit ha forma Press = a*Conteggi+b. Il grafico è AX3 
\$a = -0.00415 \pm 0.00000 \$ 
\$b = 706.59050 \pm 0.00000 \$

\begin{verbatim}
[[Fit Statistics]]
    # fitting method   = leastsq
    # function evals   = 7
    # data points      = 2
    # variables        = 2
    chi-square         = 2.000e-250
    reduced chi-square = 0.00000000
    Akaike info crit   = -1147.29255
    Bayesian info crit = -1149.90625
##  Warning: uncertainties could not be estimated:
[[Variables]]
    a: -0.00415430 +/- 0.00000000 (0.00%) (init = 1)
    b:  706.590504 +/- 0.00000000 (0.00%) (init = 1) 
\end{verbatim}

P di dimezzamento: 696.27+/-0.21, con conteggio (2.48+/-0.05)e+03 
D = 61.2+/-1.0 P standard = 1013.2+/-0 
Tstandard = 293.15+/-0 Tlab = 296.9+/-1.0 
D di dimezzamento = 41.5+/-0.7 
Rtot = Ddimezzamento + (R residuo) 1.39+/-0.10 = 0.0517+/-0.0008 

RANGE NEL MYLAR

Il fit ha forma Range = aE^2 + bE+c. Il grafico è AX4 
\$a = 0.00009 \pm 0.00001 \$ 
\$b = 0.00033 \pm 0.00013 \$ 
\$c = 0.00015 \pm 0.00030 \$

\begin{verbatim}
[[Fit Statistics]]
    # fitting method   = leastsq
    # function evals   = 9
    # data points      = 4
    # variables        = 3
    chi-square         = 45.0000000
    reduced chi-square = 45.0000000
    Akaike info crit   = 15.6814725
    Bayesian info crit = 13.8403556
[[Variables]]
    a:  9.0000e-05 +/- 1.3416e-05 (14.91%) (init = 1)
    b:  3.2700e-04 +/- 1.2760e-04 (39.02%) (init = 1)
    c:  1.5050e-04 +/- 2.9989e-04 (199.27%) (init = 1)
[[Correlations]] (unreported correlations are < 0.100)
    C(a, b) = -0.999
    C(b, c) = -0.999
    C(a, c) = 0.995 

\end{verbatim}

\begin{longtable}[]{@{}rrr@{}}
\toprule
spess & chn & u\_chn \\
\midrule
\endhead
0.9 & 1061 & 2 \\
1.4 & 1049 & 3 \\
2.8 & 1016 & 3 \\
4.2 & 981 & 8 \\
5.1 & 963 & 8 \\
\bottomrule
\end{longtable}

\begin{longtable}[]{@{}ll@{}}
\toprule
Energie misurate & Energie attese \\
\midrule
\endhead
5.387+/-0.009 & 5.390+/-0.006 \\
5.326+/-0.013 & 5.336+/-0.008 \\
5.160+/-0.013 & 5.184+/-0.022 \\
4.984+/-0.034 & 5.03+/-0.05 \\
4.893+/-0.034 & 4.93+/-0.06 \\
\bottomrule
\end{longtable}

\hypertarget{rate-in-funzione-della-distanza}{%
\section{Rate in funzione della
distanza}\label{rate-in-funzione-della-distanza}}

\begin{longtable}[]{@{}rrrrr@{}}
\toprule
d & u\_d & cnt & t & u\_t \\
\midrule
\endhead
49.8 & 0.05 & 43746 & 200 & 0.01 \\
46.65 & 0.05 & 43724 & 150.43 & 0.01 \\
39 & 0.05 & 48153 & 101.93 & 0.01 \\
31.45 & 0.05 & 45742 & 51.29 & 0.01 \\
27.3 & 0.05 & 47219 & 35.76 & 0.01 \\
23.65 & 0.05 & 48594 & 22.37 & 0.01 \\
19.75 & 0.05 & 47525 & 11.91 & 0.01 \\
16.15 & 0.05 & 43125 & 4.95 & 0.01 \\
\bottomrule
\end{longtable}

\begin{longtable}[]{@{}
  >{\raggedright\arraybackslash}p{(\columnwidth - 6\tabcolsep) * \real{0.1739}}
  >{\raggedright\arraybackslash}p{(\columnwidth - 6\tabcolsep) * \real{0.2464}}
  >{\raggedright\arraybackslash}p{(\columnwidth - 6\tabcolsep) * \real{0.3043}}
  >{\raggedright\arraybackslash}p{(\columnwidth - 6\tabcolsep) * \real{0.2754}}@{}}
\toprule
\begin{minipage}[b]{\linewidth}\raggedright
Distanze
\end{minipage} & \begin{minipage}[b]{\linewidth}\raggedright
Tempi
\end{minipage} & \begin{minipage}[b]{\linewidth}\raggedright
Counts
\end{minipage} & \begin{minipage}[b]{\linewidth}\raggedright
Rate
\end{minipage} \\
\midrule
\endhead
34.0+/-1.0 & 200.000+/-0.010 & (4.375+/-0.021)e+04 & 218.7+/-1.0 \\
30.9+/-1.0 & 150.430+/-0.010 & (4.372+/-0.021)e+04 & 290.7+/-1.4 \\
23.2+/-1.0 & 101.930+/-0.010 & (4.815+/-0.022)e+04 & 472.4+/-2.2 \\
15.6+/-1.0 & 51.290+/-0.010 & (4.574+/-0.021)e+04 & 892+/-4 \\
11.5+/-1.0 & 35.760+/-0.010 & (4.722+/-0.022)e+04 & 1320+/-6 \\
7.8+/-1.0 & 22.370+/-0.010 & (4.859+/-0.022)e+04 & 2172+/-10 \\
3.9+/-1.0 & 11.910+/-0.010 & (4.753+/-0.022)e+04 & 3990+/-19 \\
0.3+/-1.0 & 4.950+/-0.010 & (4.312+/-0.021)e+04 & (8.71+/-0.05)e+03 \\
\bottomrule
\end{longtable}

Il fit ha forma Range = a/x\^{}2 + b. Il grafico è AX5 
\$a = 1066.11126\pm 825.90998 \$ 
\$b = 326.63097 \pm 112.20730 \$

\begin{verbatim}
[[Fit Statistics]]
    # fitting method   = leastsq
    # function evals   = 12
    # data points      = 8
    # variables        = 2
    chi-square         = 131750.160
    reduced chi-square = 21958.3601
    Akaike info crit   = 81.6737691
    Bayesian info crit = 81.8326522
[[Variables]]
    a:  1066.11126 +/- 825.909981 (77.47%) (init = 1)
    b:  326.630972 +/- 112.207304 (34.35%) (init = 1) 
\end{verbatim}

Il fit ha forma Range = b\emph{0.5}(1-(4/pi)\emph{np.arcsin(1/np.sqrt(2
+ 0.5}(a/x)**2))). Il grafico è AX6 
\$a = 14.80192 \pm 0.91625 \$ 
\$b =15108.72430 \pm 1367.26672 \$

\begin{verbatim}
[[Fit Statistics]]
    # fitting method   = leastsq
    # function evals   = 24
    # data points      = 8
    # variables        = 2
    chi-square         = 2579.64832
    reduced chi-square = 429.941387
    Akaike info crit   = 50.2077345
    Bayesian info crit = 50.3666176
[[Variables]]
    a:  14.8019238 +/- 0.91624927 (6.19%) (init = 40)
    b:  15108.7243 +/- 1367.26672 (9.05%) (init = 1e+11)
[[Correlations]] (unreported correlations are < 0.100)
    C(a, b) = -0.923 
\end{verbatim}

Il fit ha forma Range = b\emph{0.5}(1-(4/pi)\emph{np.arcsin(1/np.sqrt(2
+ 0.5}(a/(x-c))**2))). Il grafico non c'è 
\$a = 10.54729 \pm 1.48403 \$
\$b = 38281.21989 \pm 12748.57322 \$ 
\$c = -3.54370 \pm 0.84905 \$

\begin{verbatim}
[[Fit Statistics]]
    # fitting method   = leastsq
    # function evals   = 68
    # data points      = 8
    # variables        = 3
    chi-square         = 516.697013
    reduced chi-square = 103.339403
    Akaike info crit   = 39.3441209
    Bayesian info crit = 39.5824455
[[Variables]]
    a:  10.5472857 +/- 1.48402813 (14.07%) (init = 40)
    b:  38281.2199 +/- 12748.5732 (33.30%) (init = 1e+11)
    c: -3.54369879 +/- 0.84904713 (23.96%) (init = 1)
[[Correlations]] (unreported correlations are < 0.100)
    C(a, b) = -0.993
    C(b, c) = -0.965
    C(a, c) = 0.933 
\end{verbatim}
